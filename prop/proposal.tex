\documentclass[a4paper]{article}
\usepackage[english]{babel}
\usepackage[hmargin=3cm,vmargin=4cm]{geometry}
\usepackage{microtype}
\usepackage{color}
\pagestyle{empty}

\begin{document}
\begin{center}
\begin{Huge}
Towards chemically stable anti-oxidation/electrical passivation layers on gallium-arsenide surfaces. \\[1cm]
\end{Huge}
\hline
\end{center}
\begin{flushright}
A proposal for a one-year Masters graduation, visiting-student research project at\newline
the Weizmann Institute of Science\newline
-- by Timo Bretten, \today
\end{flushright}
\section*{Introduction}
The field of semiconductor physics has led to incredible technological advances, virtually all of today's consumer electronics is based on silicon (Si) semiconductor technology. Apart from that other useful devices, such as the light emitting diode (LED) and other opto-electronics are also based on semiconductors, but not necessarily on Si. As a key technology solar cells offer a possibility to overcome our increasing energy demand while being balanced from a sustainability point of view. Here, gallium based III-IV semiconductors promise great potential for many already existing but also for emerging technologies. When talking about solar cells, gallium-arsenide (GaAs) based cells are especially promising because their band-gap is nearly perfectly aligned to the sun's spectrum as received on earth and can even be further engineered by additional lattice atoms such as aluminium (Al), indium (In) or phosphorous (P). However, there are also drawbacks to using GaAs, to wit, its unstable surfaces which are prone to oxidation under ambient condition and a lack of understanding of the surface chemistry involved~\cite{review}. In contrast to silicon, gallium-arsenide's natural oxides do not form a natural passivation layer, neither with respect to electrical nor to chemical passivation. This results in unwanted recombination sites within the material's band gap and consequent energy loss as well as material degradation over time. For example it has recently been shown by Martin Green, that even today's best GaAs solar cell could reach a $\sim 40\, mV$ increase in open-circuit voltage ($V_{oc}$) if non-radiative recombination were suppressed~\cite{green}. Yet, due to the material's difficulties, no sufficiently chemically stable passivation layers could be achieved by wet-chemical methods and costly techniques such as molecular-beam-epitaxy have to be employed to passivate GaAs, not at least because GaAs surface chemistry is more diverse~\footnote{At least two instead of just one lattice atom with several oxidation states each} and thus much more complicated than Si's. Thus, studying gallium-arsenide surface chemistry and surface passivation in particular is interesting for applied as well as fundamental research.
\section*{Proposed Research}
\subsection*{Short summary -- work so far}
Substantial research into GaAs surface chemistry and passivation has been carried out at the Department of Materials and Interfaces of the Weizmann Institute of Science and elsewhere. So far, (di-)carboxylic, thiol, phosphate \& phosphonate functional groups have been adsorbed to GaAs surfaces.~\cite{Shpaisman1},~\cite{Shpaisman2},~\cite{thesis}. Depending on the molecular `tail', surface passivation against oxidation, i.e. \emph{chemical passivation}, was demonstrated, whereas suppression of recombination sites within the band-gap, i.e. \emph{electrical passivation}, depends more on the headgroup that makes the actual bond with the GaAs surface. It could be shown by systematic variation of tailor made molecules that surface physics can be varied with the dipole of the molecule adsorbed to the surface~\cite{VilanNature}. Alkanethiol passivating properties could be enhanced by post-processing with ammoniumsulfide~\cite{Dubowski} and in an unrelated study it is stated that phosphorous leads to more chemically stable passivation layers since its oxide is thermodynamically more stable than sulphur-oxide with respect to arsenic~\cite{oxidpaper}.\\
In a different approach -- functionalising the GaAs surface with chlorine and subsequent nucleophilic Grignard reaction -- surface alkylation by a direct Ga-C bond was demonstrated and shown to provide fair passivating characteristics, electrically as well as chemically~\cite{Grignard}. An interesting and experimentally relatively simple mechanism for self-assembled monolayer (SAM) formation on hydrophilic surfaces has also been demonstrated~\cite{Nie}. Here, an amphiphile,  octadecylphosphonic acid (OPA), is dissolved in a nonpolar solvent. Strongly dependent on the hydrophobicity of the solvent, OPA will either aggregate at the solid-solvent interface with its headgroups pointing towards the solid or form micelles within the solvent. 
\subsection*{Outlook -- proposed work}
I propose to adopt the mechanism demonstrated by Nie for silicon-oxide($SiO_2$) and mica to gallium arsenide surfaces and investigate post-processing possibilities to answer these questions:
\begin{itemize}
\item can the mechanism demonstrated by Nie for hydrophillic surfaces be applied to GaAs and if so
	\begin{itemize}
	\item what are the reaction conditions needed for formation of high quality self assembled monolayers of different chemical species onto GaAs surfaces? 
	\item can a systematic relation between surface-molecule interaction strength, solvent polarity as indicated by their dielectric constant, $\epsilon$, and molecular polarity/amphiphilic character be extracted?
	\end{itemize}
\item can post-processing with ammoniumsulfide enhance surface passivation with chemical species other than alkanethiol?
\item will phosphorous rather than sulphur atoms adsorbed to the surface lead to even more stable passivation layers?
\end{itemize}
Since  the mechanism intrinsically rests upon amphiphilic molecules being dissolved in a non-polar solvent, this route offers a water-free way to surface passivation and can thus be explored in the inert atmosphere of a glove box, hopefully allowing for virtually no oxygen contamination on the surface prior to passivation and after chemical etching. Nie studied octadeclyphosphonic acid initially and found that successful layer formation depends strongly on the dielectric constant of the solvent used to deliver the amphiphilic molecules compared to the polarity of the surface. By varying the amphiphilic character of the phosphonic acid to be delivered, i.e. by systematically changing the length of the hydrophobic `tail', a reliable relation between solvent polarity, amphiphilic character and headgroup-surface interaction may possibly be extracted. Since oxide free GaAs' surfaces are hydrophobic rather than hydrophilic, it is reasonable to start with OPA and more hydrophobic solvents with an $\epsilon$ less than 4 , such as cyclohexane, toluene or o-xylene. Regarding the molecules to be used, it is too early to state definitively within which boundaries the tail-length should be varied. However, since GaAs is hydrophobic, I suspect molecules with shorter tails than OPA -- decylphosphonic, octylphosphonic, hexylphosphonic acid... -- to be more promising than longer ones. If the amphiphiles have too hydrophobic a tail they could develop a tendency to aggregate with their tails rather than their heads toward the surface. Once a relation as mentioned has been found, it may be possible to predict which solvents should be used to form high quality monolayers of amphiphilic molecules with suflonic and carboxylic headgroups, which are already known to bind to GaAs and provide electrical passivation. Contrary to the probable requirements for the mechanism to work, to fulfil the ultimate goal of a high quality SAM with good electrical and especially chemical passivating properties, it is likely that `long-tailed' molecules~\footnote{10 carbon-atoms and more} will produce best results. The longer the tail, the stronger the stabilising intermolecular van der Waals interactions within the monolayer will be and the more of a barrier against oxygen the monolayer will constitute. Of course, if junction formation for solar-cell device fabrication is aimed at, it should also be considered that -- at least at high forward bias -- longer-tailed monolayers will result in more insulating layers, hindering current transport across the junction~\cite{alkyltransport}. Finding the right balance between desirable chemical stability of the layer, chemical passivation against oxidation and its transport characteristics within the boundaries set by the mechanism may well be a research project of its own.\\
Once a stable SAM is achieved\footnote{Either via the mechanism proposed by Nie or, if this proves to be inapplicable to GaAs, the one already demonstrated by Shpaisman~\cite{Shpaisman1}}, post-processing possibilities may be explored, from this point on independently of the research into SAM formation chemistry. The dual approach offers a chance to obtain viable results even if experimental obstacles are encountered in one of the separate directives.\\
After preparation, both ellipsometry and contact-angle measurements provide a quick first characterisation of the monolayer: layer thickness~\footnote{and thus the angle the molecules make with the GaAs surface} and surface coverage can be gauged. A peak shift in the $CH_2$ antisymmtric stretch of (fourier-transform) infrared spectra can reveal information about the monolayer's molecules's phase -- liquid vs. solid-like -- and thus also about their stability and adherence to the GaAs surface. Contact-Potential-Difference reveals how successful the surface was passivated electrically. Ascertaining a level of electrical passivation and observing its degradation over time, for example as a function of ambient oxygen or water-vapour concentration, would gain insight into the layer's chemical stability. At the same time, X-ray Photoelectron Spectroscopy (XPS) can reveal information about the oxidation state of the atoms involved and in what ratios the different oxidation states are present. It offers the possibility to check if all the oxygen present on the surface is due to the phosphonic, sulfonic or carboxylic groups or due to surface oxides, thus, XPS measurements, especially when done over time, possibly reveal a great deal of information about chemical passivation.\\
Junctions for study of current-voltage characteristics could be made either with Hg-drop or LOFO-Au deposition, both these approaches have already been shown to form non-destructive, non-shorting top-contacts on alkyl-tailed monolayers~\cite{VilanNature},~\cite{hgdrop}. The drawback of these two materials is their being opaque, thus only dark junction characteristics could be studied. Top-contacting via a conductive polymer, for example PDOT:PSS~\cite{pdot}, or thinly evaporated lead~\cite{pbevap} using evaporation-masks could also be explored and would allow for dark as well as light characterisation of the junction. It has been hinted that surface layer quality can be assessed from I-V measurements alone~\cite{Shpaisman1}. If this trend can be reproduced, it could save considerable time in analysing the GaAs surfaces. Generally, temperature dependent I-V would gain more detailed insight into the transport mechanisms across the junction -- thermionic emission vs. tunneling vs. more involved mechanisms -- and its dependence on the type of bond on the surface.\\[10pt]
Substantial and promising research into surface chemistry \& electrical transportation across molecular monolayers in general and into gallium-arsenide passivation in particular is being carried out at the Weizmann Institute of Science. Supported by the experimental facilities and expertise present in the group, I trust the proposed strategy to provide a fruitful step towards reproducible and reliable preparation of stable anti-oxidation/chemical passivation layers on gallium-arsenide.

\newpage
\hyphenrules{nohyphenation}\exhyphenpenalty 10000
\begin{thebibliography}{9}
\bibitem{review}
	B. Schwartz (1975): 
	\emph{GaAs surface chemistry – a review}.
	C R C Critical Reviews in Solid State Sciences, 5:4,{609-624}
\bibitem{green}
	M.A. Green
	\emph{Radiative efficiency of state of the art photovoltaic cells}.
	Prog. Photovolt: Res. Appl. 2012; 20:{472-476}

\bibitem{Shpaisman1}
	Hagay Shpaisman, Eric Salomon, Guy Nesher, Ayelet Vilan, Hagai Cohen, Antoine Kahn \& David Cahen
	\emph{Electrical Transport and Photoemission Experiments of Alkylphosphonate Monolayers on GaAs}.
	J. Phys. Chem. C 2009, 113, pp.{3313 – 3321}
\bibitem{Shpaisman2}
	Guy Nesher, Hagay Shpaisman \& David Cahen
	\emph{Effect of Chemical Bond Type on Electron Transport in GaAs-Chemical Bond-Alkyl/Hg Junctions}.
	J. AM. CHEM. SOC. 2007, 129, pp.{734 - 735}
\bibitem{thesis}
	Ayelet Vilan
	\emph{M.Sc. Thesis}
	at the Weizmann Institute of Science, Israel, 1997
\bibitem{VilanNature}
  Ayelet Vilan, Abraham Shanzer \& David Cahen,
  \emph{Molecular control over Au/GaAg diodes}.
  Nature, March 2000,s 404, pp.{166 - 168}.
\bibitem{Dubowski}
	Jan J. Dubowski \emph{et. al}
	\emph{Enhanced Photonic Stability of GaAs in Aqueous Electrolyte Using Alkanethiol Self-Assembled Monolayers and Postprocessing with Ammonium Sulfide}.
J. Phys. Chem. C 2012, 116, pp.{2891 − 2895}.
\bibitem{oxidpaper}
	P. Moriarty, B. Murphy \& G. Hughes
	\emph{Scanning tunneling microscopy study of the ambient oxidation of passivates GaAs(100) surfaces}.
	J. Vac. Sci. Technol. A 11(4), Jul/Aug 1993
\bibitem{Grignard}
	Stephen Maldonado \emph{et. al}
	\emph{Wet Chemical Functionalization of III−V Semiconductor Surfaces: Alkylation of Gallium Arsenide and Gallium Nitride by a Grignard Reaction Sequence}.
	Langmuir 2012, 28, pp.{4672 − 4682}
\bibitem{Nie}
	Heng-Yong Nie, Mary J. Walzak \& N. Stwart McIntyre
	\emph{Delivering Octadecylphosphonic Acid Self-Assembled Monolayers on a Si Wafer and OtherOxide Surfaces}.
	J. Phys. Chem. B 2006, Vol. 110, No. 42, pp.{21101 - 21108}
\bibitem{alkyltransport}
	Omer Yaffe, Luc Scheres, Lior Segev, Ariel Biller, Izhar Ron, Eric Salomon, Marcel Giesbers, Antoine Kahn, Leeor Kronik, Han Zuilhof, Ayelet Vilan, David Cahen\\
	\emph{Hg/Molecular Monolayer-Si Junctions: Electrical Interplay between Monolayer Properties and Semiconductor Doping Density}.
	J. Phys. Chem. C 2010, 114, pp.{10270 – 10279}
\bibitem{hgdrop}
	G. Nesher, A. Vilan, H. Cohen, D. Cahen, F. Amy, C. Chan, J.H. Hwang, A. Kahn
	\emph{Radiation damage to alkyl chain monolayers on semiconductor substrates investigated by electron spectroscopy}.
	J. Phys. Chem. B 2006, 110, pp{21826-21832}.
\bibitem{pdot}
	Masahiro Ono, Zeguo Tang, Ryo Ishikawa, Takuya Gotou , Keiji Ueno, and Hajime Shirai
	\emph{Efficient Crystalline Si/Poly(ethylene dioxythiophene):Poly(styrene sulfonate):Graphene Oxide Composite Heterojunction Solar Cells}
	Applied Physics Express 5 (2012) 032301
\bibitem{pbevap}
	Robert Lovrincic, Olga Kraynis, Rotem Har-Lavan, Abd-Elrazek, Haj-Yahya, Wenjie Li, Ayelet Vilan, and David Cahen
	\emph{A permanent, stable, and simple top-contact for molecular electronics on Si:Pb evaporated on organic monolayers}
	Pre-published online at http://arxiv.org/pdf/1211.2174.pdf
\end{thebibliography}
\end{document}
