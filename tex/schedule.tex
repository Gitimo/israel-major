\documentclass[a4paper,10pt]{article}

\usepackage[T1]{fontenc}
\usepackage[utf8]{inputenc}
\usepackage{multicol}
\usepackage{upgreek}
\usepackage{color}
\usepackage{amsmath}
\usepackage{amsfonts}
\usepackage{amssymb}
\usepackage{graphicx}
\usepackage[bottom]{footmisc}
\usepackage{multirow}
\usepackage[english]{babel}
\usepackage{float}
\usepackage[hang]{caption}
\usepackage{subfig}
\usepackage{microtype}
\usepackage[pdftex]{hyperref}
\hypersetup{colorlinks=false,pdfborder={0 0 0}}
\usepackage[a4paper]{geometry}

\pagestyle{empty}
\begin{document}
\part*{Experiments till Pesah}
\section*{Lakeshore}
\begin{itemize}
	\item Photo-Diode measurements of light intensity: Wednesday
	\item Saturation test with alumina: Wednesday
	\item further steps depending on outcome
\end{itemize}
\section*{Nano Particles}
\begin{itemize}
	\item synthesis from cyclohexanone as solvent: 6$^{\text{th}}$ to 10$^{\text{th}}$ of April
	\item finding deposition conditions: 6$^{\text{th}}$ to 10$^{\text{th}}$ of April
	\item poling: 6$^{\text{th}}$ to 10$^{\text{th}}$ of April
\end{itemize}
Cyclohexanone is a better solvent for the polymer than acetone which is used throughout the literature. I should be able to synthesise NPs in higher concentration. Their quality and size might also be improved, because the cyclohexanone/H$_2$O+MeOH interface is much more stable than acetone/H$_2$O+MeOH. This is a new procedure and might improve the synthesis compared to what is known in the literature.\\
I will check the quality and size distribution of the NPs by SEM and will use the chance to get permission from Asaf to work  there on my own. I will also pursue Dynamic Light Scattering measurements with the Physics/Optics groups. I will use GATIR to check the crystal-phase. To deposit NPs on Si, I will set up a  \lq{}nitrogen-tent\rq{}, so the solution can evenly evaporate at room-temperature or slightly above without too much O$_2$ exposure. Spin-casting is not an option because of wetting, but draw-casting or mechanical pressure might work.\\ 
Once I have a film, I can check (unpoled) CPD \& SPV. I will try to pole in high-field, without touching the surface. If that doesn\rq{}t work, I\rq{}ll try intimate-contact poling and compare CPD \& SPV.
\section*{Polymer Layer}
\begin{itemize}
	\item retry poling without PDOT:PSS: 13$^{\text{th}}$ \& 14$^{\text{th}}$ of April
	\item poling with PDOT:PSS: 15$^{\text{th}}$ \& 16$^{\text{th}}$ of April
\end{itemize}
I want to retry poling without PDOT one last time, using 80$^{\circ}$C as poling temperature, in the glovebox. This temperature should not change the crystalline phase of the polymer and enhance alignment with the external field at the same time. If this doesn\rq{}t work, I will deposit PDOT:PSS on top of the polymer layer and try to apply the field directly through the PDOT top-layer.
\end{document}
